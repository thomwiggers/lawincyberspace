\documentclass{article}

\usepackage{hyperref}
\usepackage{microtype}

\author{Thom Wiggers\\ \small Informatica, RU \\ \small s4119444 \\ \small thom@thomwiggers.nl }
\title{Privacy By Design\\ {\large Law in Cyberspace, Assignment 3}}

\begin{document}
\maketitle

\section{Introduction}

\section{Fitness Apps}

In recent years an increasing number of fitness applications has become available across the various mobile platforms, such as Android and Apple's iOS.
These applications often collect a lot of data that many people consider sensitive, from behavioural data like how often one exercises and how much someone eats, to bodily characteristics such as weight, age and length.
Some apps even allow to monitor someones heart rate. In this paper I will be considering two apps in further depth.

Runkeeper is one such app available both on iPhones and on devices running Android. It allows users to ``track, manage and share all of [their] fitness activities''.\footnote{\emph{Terms of Service}, FitnessKeeper, Inc. \url{https://runkeeper.com/termsofservice/}}
Runkeeper allows users to record activities, their location, body measurements, weight, nutrition information, wheter they have diabetes and their sugar measurements and information about their sleep habits.\footnote{\emph{Settings -- Sharing}. \url{http://runkeeper.com/settings/sharing}}

Similarly, Fitbit\footnote{\emph{Fitbit}. \url{https://Fitbit.com}} sell a wristband and provide a companion app that allows users to track their steps, activity, heart rate, sleep cycle, location and eating habits.
Fitbit allows users to export this information to other apps, including to Runkeeper.\footnote{\emph{Fitbit API}. \url{http://dev.Fitbit.com}}

\section{Legal conditions}

\subsection{Applicable law}

Before we need consider if European privacy laws apply to both of these companies located in the United States of America.
When the Dutch privacy protection agency (\emph{College Bescherming Persoonsgegevens}, CBP) was researching popular instant messaging app WhatsApp, which is also based in the United States, they decided that they were competent to investigate the case and that Dutch privacy laws did apply.
They based this on consideration 20 with the EU Data Protection Directive 95/46/EC and on two opinions by the Article 29 Working Party, one of which I will show below.

Consideration 20 on directive 95/46/EC specifies that \emph{the fact that the processing of data is carried out by a person established in a third country must not stand in the way of the protection of individuals provided for in this Directive; whereas in these cases, the processing should be governed by the law of the Member State in which the means used are located, and there should be guarantees to ensure that the rights and obligations provided for in this Directive are respected in practice}.
This consideration is further established in article 4 of the directive.

The Article 29 Working Party considered\footnote{\emph{Working document on determining the international application of EU data protection law to personal data processing on the internet by non-EU based web sites}, 30 May 2002, Article 29 Working Party} that ``making use of equiment'' as in article 4 (1) (c) of the directive means that ``such equipment should be at the disposal of the controller, [\ldots] while at the same time it is not necessary that the controller exercise full control over the equipment''.
As the mobile phones run applications with full control over the phone, even though this is limited to when the application is running, I think this article 4 (1) (c) applies and that the app providers do make use of the smartphone as a means of data processing.

These fitness apps should thus adhere to EU privacy law.

\subsection{Roles in the context of 95/46/EC}

In both these cases there is a significant flow of personal data from the users to the app providers.
If we want to view this data in light of the Data Protection Directive\footnote{Directive 95/46/EC}, we must first establish the various roles involved in the processing of this data.
Article 2 of the directive establishes the roles of the data subject, data controller and the data processor.\footnote{95/46/EC, article 2, sub (a), (d) and (e)}

The data subject is the identifyable person whom the personal data identifies.
In this case this is clearly the user of the Runkeeper or Fitbit applications.

The controller has the capability to deterimine the purposes and the means of the processing of personal data.
In the case of these fitness apps, the app provider is the data controller as both of their terms of service grant the provider an virtually unlimited licence to process the provided user data.\footnote{\emph{Fitbit Website Terms and Conditions}. \url{http://www.Fitbit.com/uk/terms}} \footnote{\emph{Runkeeper Terms of Service}. \url{http://runkeeper.com/termsofservice}}
Runkeeper even goes as far as as not covering voluntarily provided data, which includes most if not all fitness data the users post to the service, by the limitations of the privacy policy.\footnote{\emph{Runkeeper Privacy Policy}. \url{http://runkeeper.com/privacypolicy}} \footnote{After reading this, the author decided to terminate their Runkeeper account.}

The primary data processors in this case would also be the app providers.
They process the information by providing it on their websites and through their apps.
Both their privacy policies indicate that they may use other parties for certain parts of the service, but that those parties will not be allowed to further desiminate the information.

\subsection{Lawful processing}

The Charter of Fundamental Rights of the European Union grants everyone in the Union ``the right to the protection of personal data concerning him or her'', through article 8 (1).
The second paragraph explains that personal data can only be processed if it is done ''fairly for specified purposes and on the basis of the consent of the person concerned or some other legitimate basis laid down by law.``
This establishes two criteria for lawful processing of personal data: (i) specificity and (ii) consent (or another legitimate basis defined by law).
These two criteria have been integrated in the Data Protection Directive (95/46/EC) as articles 6 and 7, respectively.

The data Fitbit and Runkeeper process can be devided in three parts. Both companies do this in their privacy policies and terms of service: they consider data collected for account administration purposes, logged data collected from devices and ``User Generated Data'' (Fitbit) or ``User Content'' (RunKeeper), which are for example activity reports or comments posted on the website.

The legal basis for the processing of the first kind of data can be found in article 7 (b) of the directive: processing of items like an email address and password is necessary to perform the services for the user.
This basis is only enough if the data is only processed for these and compatible purposes.
Further processing purposes might be compatible if these purposes are for ``historical, statistical or scientific purposes''.\footnote{95/46/EC article 6 (1) (b), second sentence}.
Other compatible purposes should be tested according to the Article 29 Working Party opinion on purpose limitation.\footnote{\emph{Opinion 03/2013 on purpose limitation}, 2 April 2013, Article 29 Working Party}
Both companies explain in their privacy policies that they might share or sell aggregated, non-identifying data.
If data is fully anonymised the privacy directive no longer applies, so this would be legitimate.\footnote{95/46/EC article 2 (a)} \footnote{\emph{Opinion 4/2007 on the concept of personal data}, 20 June 2007, Article 29 Working Party}

Processing of the logged data, for example website analytics, would require the consent of the users.
They provide this consent primarily through agreeing with the privacy policy.
Both services require users to agree with the privacy policy before users may use the service.
Article 7 (a) however requires ``unambiguous consent''.
The Article 29 Working Party requires that this consent be \emph{freely given}, \emph{specific} and \emph{informed}.\footnote{\emph{Opinion 15/2011 on the definition of consent}, 13 July 2011, Article 29 Working Party}
Notably, the Working Party states in an example about social networks that ``The user should be put in a position to give free and specific consent to receiving behavioural advertising, independently of his access to the social network service.''
This appears not to be the case with either the consent given to Fitbit or Runkeeper.

Even if someone does not have an account and thus supposedly agreed with the Terms of Service and Privacy Policy, however, both sites subject visitors to tracking cookies without asking for explicit permission.

The Data Protection Directive requires that data should be collected for ``specified, explicit and legitimate purposes and not further processed in a way incompatible with those purposes.''\footnote{95/46/EC article 6 (1) (b)}
This appears not to be the case with Runkeeper.
Runkeeper has an exclusion clause in its privacy policy, which excludes all ``User Content'', content users voluntarily submit to the service, from the protections the pivacy policy gives.
Runkeeper states such information ``shall be deemed to be non-confidential and we shall be free to reproduce, use, disclose and distribute such Unsollicited information to others without limitation or attribution''.\footnote{\emph{Runkeeper Privacy Policy}. \url{http://runkeeper.com/privacypolicy}}
The terms of service allow Runkeeper to do virtually anything with any part of this information unlimitedly.
This unlimitedness makes the purposes unspecified and thus makes the collection of data unlawful. Having user consent does not legitimise this, because article 6 and 7 of the directive are cumulative and must both be met simultaneously.\footnote{\emph{Opinion 03/2013 on purpose limitation}, 2 April 2013, Article 29 Working Party.}

Fitbit explain in their privacy policy various ways they may collect data and how they use data.
Fitbit allow users to select with whom they want to share certain data and they refer to this in their terms of service.
Data you mark to be shared publicly however means, according to their terms of service, that ``you hereby waive any rights of publicity and privacy with respect to the User Generated Content and any other legal or moral rights that might preclude Fitbit's use of the User Generated Content or require your permission for Fitbit to use the User Generated Content.''\footnote{\emph{User-Generated Content, Fitbit Website Terms of Service}. \url{http://www.Fitbit.com/uk/terms}}
Fitbit also, like Runkeeper, claims that by providing user-generated content users grant Fitbit a virtually unlimited licence to do whatever they like.

In conclustion, it seems that neither Fitbit or Runkeeper respect the specificity-requirement of directive 95/46/EC.

\subsection{Article 8}

A special consideration needs to be made for the information processed by these apps, because these apps process medical data.
These apps for instance process weight, body measurements or heart rate information.
Runkeeper may also track if someone is diabetic.
95/46/EC Article 8 (1) prohibits the processing of certain special categories of data, including medical data.
In this case this processing would only be allowed if the user has made the data either publicly available (for which both apps provide means) or based on explicit consent.\footnote{95/46/EC Article 8 (2) (e) and (a) respectively}
If the user marks this information as public when shared through the Runkeeper or Fitbit apps, we can consider article 8 satisfied.
If we however decide article 8 (1) does not apply because of explicit consent, we can again apply the criteria as described in the previous section and find that the consent is not adequate.

\section{PDM $\oplus$ profile transparancy tool}


\subsection{design}

\subsection{answers}

Which legal problems are solved

Which legal problems are not solved.

which new problems are created by this design.

\section{Conclusions}

\end{document}

% vim: set lbr formatoptions+=l wrap tw=0 :
