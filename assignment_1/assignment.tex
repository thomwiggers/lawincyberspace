\documentclass{article}

\usepackage{hyperref}
\usepackage{microtype}

\author{Thom Wiggers\\ \small BSc Informatica, RU \\ \small s4119444 \\ \small thom@thomwiggers.nl }
\title{Big Tax Data\\ {\large Law in Cyberspace, Assignment 1}}

\begin{document}
\maketitle

Recently there has been a lot of attention for the information hunger of the
Dutch Tax and Customs Administration (Belastingdienst). In an effort to fight
tax fraud, the agency has gradually received more and more rights to investigate
people and has started to collect more and more information.

A recent example is the case where SMSParking B.V., a company providing services
that allow customers to pay for parking space via text message or over the
internet, was asked by the Belastingdienst to hand over its 2012 administration
of who had parked where. The agency wanted to use this information to check
other people's tax forms. They asked for all the information with a vague
motivation, citing many tax areas only loosely related to where cars are.

The tax auditor based its request on article 53 of the Algemene wet inzake
rijksbelastingen (AWR). The article specifies that anyone who is obligated to
keep a tax administration as specified in article 52 AWR, can be asked to hand
over information and data which could be relevant for taxation purposes (art.~53
jo art.~47 AWR). 

SMSParking thought this request was in conflict with its obligations to
customers based on privacy law (mainly the Wet Bescherming Persoonsgegevens) and
article 8 of the European Convention on Human Rights (ECHR). They refused to
comply with the request and the Belastingdienst took them to court over this. In
first instance\footnote{ECLI:NL:RBOBR:2013:6553} the court rejected the request
of the tax agency. The tax agency already had other means to check where cars
are, namely filming number plates, and therefor this request did not comply with
the subsidiarity principle: because there already was another way to get this
data the exemption in art.~8.2 ECHR, that allows interjection in the private
affairs of people based on the economic well-being of a country, did not apply
(consideration 4.23). The judge even went as far as calling the request a
``fishing expedition''.

In the appeal of the Belastingdienst against this verdict the higher court
however threw the argument out\footnote{ECLI:NL:GHSHE:2014:2803}. The court
argued that the amount of data concerned was not a problem, but more importantly
the tax agency did not even need to know if the data was relevant for taxation
purposes (consideration 3.5.2).

This case is exemplary for the kind of data the tax agency is collecting. It
stores information like this for several years. It is easy to see that the
agency is sitting on a mountain of data.

The legislator saw this as an opportunity and recently introduced a new
law\footnote{``Besluit van 1 september 2014 tot wijziging van het Besluit SUWI
in verband met regels voor fraudeaanpak door gegevensuitwisselingen en het
effectief gebruik van binnen de overheid bekend zijnde gegevens met inzet
van SyRI''
\url{https://zoek.officielebekendmakingen.nl/dossier/33579/stb-2014-320}}.
This law allows the tax agency and other government organisations to
exchange a lot of different kinds of information, including information
about employment, fines, sex, debts, benefits, property and criminal record.
The goal of this new law, called SyRI, is to prevent fraud. When the
\emph{Raad van State}, the advisory organisation when a new law is made,
reported that the law was a huge breach of privacy, allowing as much
information as possible to be processed, they were ignored.

One of the core principles of this law is profiling: people are audited on the
statistical probability that they will commit fraud. This kind of profiling
conflicts with the presumption of innocence (e.g. art. 14.2 ICCPR, art. 6.2
ECHR).

SyRI also allows the government organisations involved to set their own targets.
This is a privilege not even the intelligence agencies are trusted with: if they
want to to go on a similar ``fishing expedition'' in wiretaps, art.~27.5 Wet op
de inlichtingen- en veiligheidsdiensten 2002 (Wiv) requires them to ask
permission to the minister to monitor keywords. They also need to target
specific persons (art.~13 Wiv).

Furthermore, the tax agency is not subject to the same kind of monitoring as the
police or intelligence agencies. All together, it is easy to see that the tax
agency has gained very far reaching tools to execute its tasks in recent years.

Perhaps this would be appropriate, if the information was kept only for taxation
purposes. In principle the tax agency has to keep information it collects
a secret (art.~67 AWR), however there is a very extensive list of exemptions.
This list, laid out in art.~43c of the Uitvoeringsregeling AWR 1994, contains,
among others; the intelligence agencies, the police and criminal prosecutors.

This means there is a back door in the legal system: the Belastingdienst can be
used to access information the police and other agencies aren't allowed to have.
This suggests a breach of art.~1 Wetboek van Strafvordering, which states that
prosecution may only happen in the ways designated by the law.

The data hunger of the tax agency should be severely limited and profiling
should be stopped. In addition, the tax agency should be prohibited to share
information with the police that is not relevant to any tax fraud. This should
stop the overreaching of art.~8.2 ECHR and better protect the privacy of
citizens involved.

\section*{Sources}

Other than the in-line cited legal sources, I used the following sources:

\begin{itemize}

    \item Privacy Barometer, \emph{Doorlichten burgers sociale zekerheid}.
        \url{http://www.privacybarometer.nl/maatregel/85/De_transparante_burger}

    \item Maurits Martijn, \emph{Vergeet de politiestaat. Welkom in de 
        belastingstaat.} 
        \url{https://decorrespondent.nl/1766/Vergeet-de-politiestaat-Welkom-in-de-belastingstaat/134502282728-dd8dad73}

    \item Maurits Martijn, \emph{Politie en inlichtingendiensten kunnen via een
        achterdeur bij gegevens van de Belastingdienst}.
        \url{https://decorrespondent.nl/1841/De-Belastingdienst-als-databank-voor-OM-Politie-en-Inlichtingendiensten/56621796-863e993b}

\end{itemize}

\end{document}
