\documentclass{article}

\usepackage{hyperref}
\usepackage{microtype}

\author{Thom Wiggers\\ \small BSc Informatica, RU \\ \small s4119444 \\ \small thom@thomwiggers.nl }
\title{Privacy By Design\\ {\large Law in Cyberspace, Assignment 3}}

\begin{document}
\maketitle

\section{Introduction}

\section{WhatsApp}

Instant messaging services offer real-time communication in the form of texttransmission over the internet. While they have been around since before therise of the internet, the largest instant messaging platforms are now on mobile, with whatsapp having more than 600 million users.\footnote{\emph{WhatsApp hits 600 Million active users, Founder says}, 25 August 2014, Forbes. \url{http://www.forbes.com/sites/parmyolson/2014/08/25/whatsapp-hits-600-million-active-users-founder-says/}}

\subsection{Data involved with using WhatsApp}

WhatsApp allows users to send each other messages, pictures, videos and audio messages. Users can also share their location with their converational partner.WhatsApp also supports group chats, in which the messages are shared with everyone participating.\footnote{\emph{WhatsApp :: Home}. \url{https://www.whatsapp.com}}

WhatsApp collects the contacts a user has in their phonebook from the mobile phone. It uploads that data twice a day to the WhatsApp servers. WhatsApp uses this information to show who of the user's contacts have also installed WhatsApp and are available to be contacted. WhatsApp claims that it only collects the phone numbers from the address book. WhatApp stores both the phone numbers for in-network (i.e., subscribers) and out-of-network numbers collected from the address books, although the out-of-network numbers are hashed.\footnote{\emph{Report of Findings, Investigation into the personal information handling practices of WhatsApp Inc}, points 25-28, 15 January 2013, Office of the Privacy Commissioner of Canada} \footnote{\emph{WhatsApp :: Legal}. \url{https://www.whatsapp.com/legal/}}

Text messages sent via WhatsApp are deleted after they are delivered to the user's device.\footnote{\emph{Report of Findings, Investigation into the personal information handling practices of WhatsApp Inc}, point 77, 15 January 2013, Office of the Privacy Commissioner of Canada} \footnote{\emph{The Information WhatsApp Does Not Collect, Privacy Notice, WhatsApp :: Legal}. \url{https://www.whatsapp.com/legal/\#Privacy}} Multimedia messages such as photos, videos and audio files are however uploaded to a WhatsApp server and a link to the content (along with a thumbnail for photos and videos) is sent to the recipient(s) who can then download the media.\footnote{\emph{WhatsAPI/README.md}, 15 March 2014, \url{https://github.com/venomous0x/WhatsAPI/blob/master/README.md\#multimedia-message-sending}}.

Users can set a status message, for example ``In a meeting''. These status messages are displayed to every user of WhatsApp's services who have the phone number of the user who set a status message in their phonebooks. There is one exemption: a user can put phone numbers on a block list. Users on someone's block list can not see the status updates posted by that person. Every time the WhatsApp app is brought to the foreground, it will report that to WhatsApp, who use that information to provide ``Last seen''-information to users similarly to how status messages are shared.\footnote{\emph{Privacy Notice, WhatsApp :: Legal}. \url{https://www.whatsapp.com/legal/\#Privacy}}

WhatsApps servers in addition log information such as timestamps and sending and receiving phonenumbers whenever a message is sent or received or if status information is requested or updated.\footnote{\emph{Privacy Notice, WhatsApp :: Legal}. \url{https://www.whatsapp.com/legal/\#Privacy}}

\section{Legal conditions}



\section{PDM $\oplus$ profile transparancy tool}

\subsection{design}

\subsection{answers}

Which legal problems are solved

Which legal problems are not solved.

which new problems are created by this design.

\section{Conclusions}

\end{document}

% vim: set lbr formatoptions+=l wrap tw=0 :
